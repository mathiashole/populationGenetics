% Options for packages loaded elsewhere
\PassOptionsToPackage{unicode}{hyperref}
\PassOptionsToPackage{hyphens}{url}
\PassOptionsToPackage{dvipsnames,svgnames,x11names}{xcolor}
%
\documentclass[
  12pt,
  letterpaper,
  DIV=11,
  numbers=noendperiod]{scrartcl}

\usepackage{amsmath,amssymb}
\usepackage{iftex}
\ifPDFTeX
  \usepackage[T1]{fontenc}
  \usepackage[utf8]{inputenc}
  \usepackage{textcomp} % provide euro and other symbols
\else % if luatex or xetex
  \usepackage{unicode-math}
  \defaultfontfeatures{Scale=MatchLowercase}
  \defaultfontfeatures[\rmfamily]{Ligatures=TeX,Scale=1}
\fi
\usepackage[]{libertine}
\ifPDFTeX\else  
    % xetex/luatex font selection
    \setmonofont[scaled=.95]{inconsolata}
\fi
% Use upquote if available, for straight quotes in verbatim environments
\IfFileExists{upquote.sty}{\usepackage{upquote}}{}
\IfFileExists{microtype.sty}{% use microtype if available
  \usepackage[]{microtype}
  \UseMicrotypeSet[protrusion]{basicmath} % disable protrusion for tt fonts
}{}
\makeatletter
\@ifundefined{KOMAClassName}{% if non-KOMA class
  \IfFileExists{parskip.sty}{%
    \usepackage{parskip}
  }{% else
    \setlength{\parindent}{0pt}
    \setlength{\parskip}{6pt plus 2pt minus 1pt}}
}{% if KOMA class
  \KOMAoptions{parskip=half}}
\makeatother
\usepackage{xcolor}
\usepackage[top=2cm,bottom=2cm,left=2cm,right=2cm]{geometry}
\setlength{\emergencystretch}{3em} % prevent overfull lines
\setcounter{secnumdepth}{5}
% Make \paragraph and \subparagraph free-standing
\makeatletter
\ifx\paragraph\undefined\else
  \let\oldparagraph\paragraph
  \renewcommand{\paragraph}{
    \@ifstar
      \xxxParagraphStar
      \xxxParagraphNoStar
  }
  \newcommand{\xxxParagraphStar}[1]{\oldparagraph*{#1}\mbox{}}
  \newcommand{\xxxParagraphNoStar}[1]{\oldparagraph{#1}\mbox{}}
\fi
\ifx\subparagraph\undefined\else
  \let\oldsubparagraph\subparagraph
  \renewcommand{\subparagraph}{
    \@ifstar
      \xxxSubParagraphStar
      \xxxSubParagraphNoStar
  }
  \newcommand{\xxxSubParagraphStar}[1]{\oldsubparagraph*{#1}\mbox{}}
  \newcommand{\xxxSubParagraphNoStar}[1]{\oldsubparagraph{#1}\mbox{}}
\fi
\makeatother


\providecommand{\tightlist}{%
  \setlength{\itemsep}{0pt}\setlength{\parskip}{0pt}}\usepackage{longtable,booktabs,array}
\usepackage{calc} % for calculating minipage widths
% Correct order of tables after \paragraph or \subparagraph
\usepackage{etoolbox}
\makeatletter
\patchcmd\longtable{\par}{\if@noskipsec\mbox{}\fi\par}{}{}
\makeatother
% Allow footnotes in longtable head/foot
\IfFileExists{footnotehyper.sty}{\usepackage{footnotehyper}}{\usepackage{footnote}}
\makesavenoteenv{longtable}
\usepackage{graphicx}
\makeatletter
\newsavebox\pandoc@box
\newcommand*\pandocbounded[1]{% scales image to fit in text height/width
  \sbox\pandoc@box{#1}%
  \Gscale@div\@tempa{\textheight}{\dimexpr\ht\pandoc@box+\dp\pandoc@box\relax}%
  \Gscale@div\@tempb{\linewidth}{\wd\pandoc@box}%
  \ifdim\@tempb\p@<\@tempa\p@\let\@tempa\@tempb\fi% select the smaller of both
  \ifdim\@tempa\p@<\p@\scalebox{\@tempa}{\usebox\pandoc@box}%
  \else\usebox{\pandoc@box}%
  \fi%
}
% Set default figure placement to htbp
\def\fps@figure{htbp}
\makeatother

\KOMAoption{captions}{tableheading}
\makeatletter
\@ifpackageloaded{caption}{}{\usepackage{caption}}
\AtBeginDocument{%
\ifdefined\contentsname
  \renewcommand*\contentsname{Table of contents}
\else
  \newcommand\contentsname{Table of contents}
\fi
\ifdefined\listfigurename
  \renewcommand*\listfigurename{List of Figures}
\else
  \newcommand\listfigurename{List of Figures}
\fi
\ifdefined\listtablename
  \renewcommand*\listtablename{List of Tables}
\else
  \newcommand\listtablename{List of Tables}
\fi
\ifdefined\figurename
  \renewcommand*\figurename{Figure}
\else
  \newcommand\figurename{Figure}
\fi
\ifdefined\tablename
  \renewcommand*\tablename{Table}
\else
  \newcommand\tablename{Table}
\fi
}
\@ifpackageloaded{float}{}{\usepackage{float}}
\floatstyle{ruled}
\@ifundefined{c@chapter}{\newfloat{codelisting}{h}{lop}}{\newfloat{codelisting}{h}{lop}[chapter]}
\floatname{codelisting}{Listing}
\newcommand*\listoflistings{\listof{codelisting}{List of Listings}}
\makeatother
\makeatletter
\makeatother
\makeatletter
\@ifpackageloaded{caption}{}{\usepackage{caption}}
\@ifpackageloaded{subcaption}{}{\usepackage{subcaption}}
\makeatother

\usepackage{bookmark}

\IfFileExists{xurl.sty}{\usepackage{xurl}}{} % add URL line breaks if available
\urlstyle{same} % disable monospaced font for URLs
\hypersetup{
  pdftitle={Second},
  pdfauthor={Mathias},
  colorlinks=true,
  linkcolor={blue},
  filecolor={Maroon},
  citecolor={Blue},
  urlcolor={darkblue},
  pdfcreator={LaTeX via pandoc}}


\title{Second}
\author{Mathias}
\date{28 November, 2025}

\begin{document}
\maketitle
\begin{abstract}
Genética de poblaciones 2025
\end{abstract}

\renewcommand*\contentsname{Table of contents}
{
\hypersetup{linkcolor=}
\setcounter{tocdepth}{3}
\tableofcontents
}

Tests y Pruebas a Utilizar

Para responder a las preguntas, utilizaríamos los siguientes tests
basados en los estadísticos proporcionados:

Test \(D\) de Tajima: Para detectar desviaciones del modelo neutral en
cada área geográfica (bosque/pradera).
\(D= \frac{\pi - \phi}{varianza(\pi - \phi)}\)\hspace{0pt}\hspace{0pt}

Estadísticos F (AMOVA): Para cuantificar la estructuración genética a
diferentes niveles. \(F_{CT}\)\hspace{0pt} mide la diferenciación entre
el Bosque y la Pradera. \(F_{CT}\)\hspace{0pt} mide la diferenciación
entre subpoblaciones dentro de cada área.

A. Escenario de Expansión Geográfica

\begin{enumerate}
\def\labelenumi{\arabic{enumi}.}
\tightlist
\item
  Inferencias Demográficas a partir de los Datos Promedio
\end{enumerate}

Comparamos los estadísticos en las dos áreas geográficas para inferir la
historia demográfica:

\begin{longtable}[]{@{}
  >{\raggedright\arraybackslash}p{(\linewidth - 6\tabcolsep) * \real{0.1543}}
  >{\raggedright\arraybackslash}p{(\linewidth - 6\tabcolsep) * \real{0.0864}}
  >{\raggedright\arraybackslash}p{(\linewidth - 6\tabcolsep) * \real{0.1111}}
  >{\raggedright\arraybackslash}p{(\linewidth - 6\tabcolsep) * \real{0.6481}}@{}}
\toprule\noalign{}
\begin{minipage}[b]{\linewidth}\raggedright
Estadístico
\end{minipage} & \begin{minipage}[b]{\linewidth}\raggedright
Bosque
\end{minipage} & \begin{minipage}[b]{\linewidth}\raggedright
Pradera
\end{minipage} & \begin{minipage}[b]{\linewidth}\raggedright
Interpretación
\end{minipage} \\
\midrule\noalign{}
\endhead
\bottomrule\noalign{}
\endlastfoot
Diversidad (π) & Alto (0.050) & Bajo (0.025) & El Bosque tiene el doble
de diversidad que la Pradera. \\
Sitios Segregantes (θW\hspace{0pt}) & Alto (0.053) & Muy Bajo (0.010) &
La Pradera tiene una cantidad de alelos raros extremadamente baja. \\
D de Tajima & D≈0 & D≪0 & El Bosque está cerca del equilibrio. La
Pradera tiene un D fuertemente negativo
(D≈(0.025−0.010)/Var\textgreater0). \\
Estructura (FCT\hspace{0pt}) & \multicolumn{2}{,c,}{0.100} & & \\
\end{longtable}

\begin{enumerate}
\def\labelenumi{\arabic{enumi}.}
\setcounter{enumi}{1}
\tightlist
\item
  Proponer un Escenario de Expansión (Hipótesis)
\end{enumerate}

El patrón es consistente con una expansión demográfica reciente y rápida
(Cuello de Botella) que ocurrió en la Pradera.

Hipótesis: La especie se originó o mantuvo una población grande y
estable en el Bosque (evidenciado por D≈0). Desde el Bosque, hubo un
evento de colonización o expansión hacia la Pradera. Este evento de
colonización implicó un fuerte cuello de botella fundador.

\begin{verbatim}
El cuello de botella en la Pradera redujo drásticamente la diversidad (π bajo).

El crecimiento posterior no ha dado tiempo suficiente para que se acumulen nuevos polimorfismos, llevando a un exceso de alelos de baja frecuencia (o pocos alelos en general, haciendo θW​ muy bajo) y un D fuertemente negativo.
\end{verbatim}

\begin{enumerate}
\def\labelenumi{\arabic{enumi}.}
\setcounter{enumi}{2}
\tightlist
\item
  Puesta a Prueba de la Hipótesis Demográfica
\end{enumerate}

La forma más robusta de poner a prueba esta hipótesis es el Análisis de
Coalescencia y Reconstrucción Demográfica (e.g., usando el SFS o PSMC):

\begin{verbatim}
Estimación de Ne​ en el Tiempo: Comparar las estimaciones del tamaño efectivo de población (Ne​) en el pasado entre las dos áreas. Se esperaría que la Pradera muestre un Ne​ ancestral mucho más pequeño que el Bosque, seguido de un crecimiento exponencial.

Ajuste del SFS (Espectro de Frecuencias de Sitios): Ajustar modelos demográficos (expansión, cuello de botella, migración) a la forma del SFS observado en la Pradera. El SFS de la Pradera debería ajustarse bien a un modelo de crecimiento rápido.

Prueba HKA (Hudson, Kreitman y Aguadé): Aplicar la prueba HKA a múltiples loci neutrales para confirmar que el patrón de baja π y θW​ se observa en todo el genoma de la Pradera, lo que confirmaría una causa demográfica global, no solo selección local.
\end{verbatim}

B. Predicciones para los Genes Candidatos (Selección)

Ahora analizamos cómo la selección local modificará estos estadísticos,
considerando que el fondo demográfico de la Pradera es un D negativo
(expansión/cuello de botella) y el del Bosque es D neutro (equilibrio).

Gen 1: Selección Direccional en la Pradera (Pelajes Claros)

Este escenario describe una Barrida Selectiva (Selective Sweep) en la
Pradera debido a un alelo ventajoso (pelaje claro).

Estadístico Pradera (Gen 1) Justificación π ≈0 (Extremadamente bajo) La
barrida selectiva elimina la variación preexistente en la región,
dejando la diversidad cercana a cero. θW\hspace{0pt} ≈0 (Extremadamente
bajo) Por la misma razón, todos los polimorfismos se eliminan. D de
Tajima D≪0 El valor negativo demográfico de la Pradera se exacerba. La
falta casi total de diversidad local crea un D de Tajima mucho más
negativo que el promedio genómico de la Pradera. LD (Desequilibrio de
Ligamiento) Alto Habrá un haplotipo extendido y de alta frecuencia en la
región del Gen 1, la ``huella'' de la mutación favorable que se fijó
rápidamente. FSC\hspace{0pt} (Local) Bajo Si la selección fijó el alelo
en todas las subpoblaciones de la Pradera, la diferenciación entre
subpoblaciones localmente en este gen (FSC\hspace{0pt}) será muy baja o
nula.

Gen 2: Selección Equilibradora en Ambos Ambientes (Regulación,
Concentración de Orina)

La selección equilibradora (balanceadora) mantiene múltiples alelos en
frecuencias intermedias, aumentando la diversidad localmente.

Estadístico Bosque y Pradera (Gen 2) Justificación π Alto (Mayor que el
promedio genómico) La selección mantiene la diversidad. La diversidad
por pares (π) se infla debido a los alelos de frecuencia intermedia.
θW\hspace{0pt} Menor (Relativamente) Los alelos se mantienen en
frecuencias medias, en lugar de acumularse como alelos raros. D de
Tajima D\textgreater0 El exceso de alelos de frecuencia intermedia
genera un D de Tajima positivo. Este D\textgreater0 sería una anomalía
en la Pradera (cuyo fondo es D\textless0) y una desviación en el Bosque
(cuyo fondo es D≈0). LD (Desequilibrio de Ligamiento) Bajo El
polimorfismo es antiguo. El tiempo y la recombinación han descompuesto
las asociaciones, manteniendo sólo el sitio bajo selección en alta
diversidad. FCT\hspace{0pt} (Local) Bajo Si la selección equilibradora
favorece el mismo polimorfismo en ambos ambientes, la diferenciación
genética entre el Bosque y la Pradera en este gen será menor que el
promedio genómico (FCT\hspace{0pt}=0.100).

Este test es altamente relevante para la parte (a) del ejercicio, ya que
fue diseñado específicamente para detectar y determinar la dirección de
una expansión geográfica basándose en datos genéticos, que es justamente
la hipótesis que proponemos para la especie de ratón.

Uso del Test de Peter y Slatkin (Ψ)

Propósito del Test

El índice Ψ (Psi) fue desarrollado por Peter y Slatkin (2013) para
detectar asimetrías en el espectro de frecuencias alélicas entre pares
de poblaciones. Estas asimetrías son la firma genética de una expansión
de rango o colonización secuencial (efectos fundadores) .

¿Cómo se Aplicaría al Problema del Ratón?

En la Parte (a), inferimos que la expansión fue del Bosque (población
ancestral más grande) a la Pradera (población cuello de
botella/fundadora).

\begin{verbatim}
Cálculo: Se calcularía el índice Ψ utilizando los SNPs genómicos entre pares de subpoblaciones.

Resultado Esperado: Si la expansión ocurrió del Bosque a la Pradera, el valor de Ψ se espera que sea positivo al calcularlo en la dirección Bosque → Pradera.

Ventaja sobre FST​: Ψ es más sensible para detectar expansiones recientes que el FST​ o las clines de heterocigosidad, ya que se centra en el cambio de frecuencia de alelos raros que se pierden durante los eventos fundadores secuenciales.
\end{verbatim}

Integración con los Estadísticos Existentes

Aunque el test Ψ no se puede calcular directamente con π y
θW\hspace{0pt} (requiere el Espectro de Frecuencias de Sitios completo),
los resultados del ejercicio ya sugieren la dirección de la expansión,
reforzando la necesidad de aplicar Ψ:

Estadístico,Bosque (Fuente),Pradera (Fundadora),Conclusión D de
Tajima,≈0,≪0,``El Bosque está en equilibrio, la Pradera ha sufrido una
expansión o cuello de botella (dirección del flujo genético).''
Diversidad (π),Alto (0.050),Bajo (0.025),La pérdida de diversidad es
característica de una población recién fundada o que sufrió un cuello de
botella. Estructura (FCT\hspace{0pt}),\multicolumn{2}{,c,}{0.100}




\end{document}
